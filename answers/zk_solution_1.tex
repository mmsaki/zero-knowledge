\documentclass{article}
\usepackage{graphicx}
\usepackage[utf8]{inputenc}
\usepackage{amsmath}
\usepackage{amsfonts}
\usepackage{hyperref}

\title{Zero Knowledge Proofs: Homework 1}
\author{Meek Msaki}
\date{February 9, 2023}

\begin{document}

\maketitle

\section*{Question 1}
$S=\mathbb{Z}_7$
\begin{itemize}
    \item[a)] $4+4 \equiv 1 \mod 7$. Answer is 1.
    \item[b)] $3\cdot5 \equiv 1 \mod 7$. Answer is 1.
    \item[c)] $3^{-1} \equiv 3^{7-2} \mod 7 \equiv 243 \mod 7 \equiv 5 \mod 7$. Verify $3\cdot5 \mod7 \equiv 1 \mod 7$. Using Fermat's little theorem, answer is 5.
\end{itemize}

\section*{Question 2}
Answer is Yes. $(\mathbb{Z}_7, +)$ is a group, it has a set of elements ${0,1,2,3,4,5,6}$ plus an operator $+$. 

\begin{itemize}
    \item[1.]It is closed. For all $a,b \in \mathbb Z_7$, the results of the operation is also in $\mathbb Z_7$.
    \item[2.]Associativity, for all $a,b,c \in \mathbb Z_7$ $(a+b)+c=a+(b+c)$
    \item[3.]There is an identity element $e$, which for each element $a\in \mathbb{Z}_7$  $a+e=e+a=a$, where the identity element $e=0$.
    \item[4.]There exists an inverse element $e$, for $a \in \mathbb Z_7$, such that $a+(-a)=(-a)+a=e$, where the inverse identity element $e=0$.
\end{itemize} 

\section*{Question 3}
$-13 \mod 5 \equiv 2 \mod 5$. Answer is 2.

\section*{Question 4}

We can simplify our function $f(x) = x^{3}-x^{2}+4x-12$ to factors $f(x)= (x-2)\cdot (x^{2}+x+6)$

Using Polynomial Remainder Theorem $\frac{P(x)}{x-a}\xrightarrow{} r=P(a)$ where $P(x) = x^{3}-x^{2}+4x-12$. If we divided $P(x)$ with a first degree polynomial $(x-a)$, and we don't get a remainder $r=0$, this verifies $(x-a)$ is a valid factor of our polynomial $P(x)$. We can solve this with our factor $(x-2)$ for $P(a)$. When $x=2$, $P(2)=2^{3}-2^{2}+4\cdot2-12=0$. Solving for $x$ using $(x-2)=0$ or $(x^{2}+x+6)=0$. Answer is $2$. The degree of this polynomial is $3$.

See: \href{https://www.khanacademy.org/math/algebra2/x2ec2f6f830c9fb89:poly-div/x2ec2f6f830c9fb89:remainder-theorem/v/polynomial-remainder-theorem-to-test-factor}{Remainder theorem: checking factors}

\end{document}
