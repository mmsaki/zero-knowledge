\documentclass{article}
\usepackage{graphicx}
\usepackage[utf8]{inputenc}
\usepackage{amsmath}
\usepackage{amsfonts}
\usepackage{hyperref}

\title{Zero Knowledge Proofs: Homework 2}
\author{Meek Msaki}
\date{February 9, 2023}

\begin{document}
\maketitle

\section*{Question 1}
Modular arithmetic
\begin{itemize}
    \item [1.]Answer is True. All odd squares are $\equiv1\mod8$ 
    \item [2.]All even square are either $\equiv{0}\mod{8}$ or $\equiv4\mod8$.
\end{itemize}

\section*{Question 2}
Generated Ethereum address ending with word \textbf{CaFe}: 

Public Key: \text{0xb2e3d94823116e9dAAd56cD95f654a1BE6e4\textbf{CaFe}}.

\section*{Question 3}
\begin{itemize}
    \item [1.]$O(n)$ means that, as the size of our input $n$ increases, in time complexity, the time it takes for our program to find a solution grows linearly. In Space Complexity, the size $n$ represents the space in memory that our program needs to run the computation.
    \item [2.]$O(1)$ means that, as the size of our input $n$ increases, the time it takes for our program to find a solution remains constant. 
    \item[3.]$O(\log{n})$ means that, as that size of our input $n$ increases, the time it takes for our program to find a solution gradually decelerates. Our input $n$ can grow exponentially while the times our program takes grows slowly compared to the size of our input.  
\end{itemize}

\section*{For proof size, which of these do we want?}
$O(\log{n})$, it's advantage is that while our input grows larger the size of our proofs remain relatively small.

\end{document}

